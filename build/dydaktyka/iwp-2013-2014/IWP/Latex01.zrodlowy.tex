%& --translate-file=cp1250pl
\documentclass[oneside,12pt,a4paper,titlepage]{report}  
\usepackage{polski}      
\usepackage{graphicx}  
\usepackage{amsmath,amssymb,amsthm,amsfonts,bbm} 
\newtheorem{tw}{Twierdzenie}


\begin{document}

MIEJSCE NA OBRAZEK Z PODPISEM

\begin{tw}
Niech ??, gdzie ?? i ?? s� pocz�tkowym i ko�cowym indeksem pewnego przedzia�u obserwowanych element�w. Przez ?? oznaczymy prawdopodobie�stwo wyst�pienia dok�adnie ?? kandydat�w w przedziale obserwacji ??. Wtedy 
\\??\\
gdzie ?? jest ustalone, a ?? i ?? s� funkcjami takimi, �e \\?? oraz ??.
\end{tw}
\begin{proof}
Dow�d przeprowadzimy indukcyjnie wzgl�dem ??. \\
\begin{enumerate}
\item ??. Na przedziale od ?? do ?? nie ma �adnych kandydat�w, je�eli najwi�kszy z element�w w�r�d pierwszych ?? wyst�puje przed momentem ??. Oznacza to, �e ??\\
\item Za��my prawdziwo�� tezy dla ??, tzn. ??.\\
\item  Niech ??. Element ?? jest ostatnim, ?? kandydatem w przedziale od ?? do ?? je�eli:
\begin{enumerate}
\item jest najwi�kszy w�r�d pierwszych ?? element�w;
\item w�r�d ??? jest dok�adnie ?? kandydat�w.
\end{enumerate}
Prawdopodobie�stwo pierwszego zdarzenia wynosi ??, a drugiego, na mocy za�o�enia indukcyjnego:
\\??\\
Zatem 
\\??\\
Ca�kuj�c przez cz�ci otrzymujemy, �e
\\??\\
\end{enumerate}
\end{proof}

\end{document}
